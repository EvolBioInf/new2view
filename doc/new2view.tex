\documentclass[a4paper, english]{article}
\usepackage{graphics,eurosym,latexsym}
\usepackage{listings}
\lstset{columns=fixed,basicstyle=\ttfamily,numbers=left,numberstyle=\tiny,stepnumber=5,breaklines=true}
\usepackage{pst-all}
\usepackage{algorithmic,algorithm}
\usepackage{times}
\usepackage{babel}
\usepackage[nodayofweek]{datetime}
\usepackage[round]{natbib}
\bibliographystyle{plainnat}
\oddsidemargin=0cm
\evensidemargin=0cm
\textwidth=16cm
\textheight=23cm
\begin{document}

\title{\texttt{new2view} \input{version}: Visualize Phylogenies}
\author{Bernhard Haubold\ignorespaces
}
\input{date}
\date{\displaydate{tagDate}}
\maketitle

\section{Introduction}
The program \texttt{new2view} draws phylogenetic trees specified in
the ``Newick'' format. This is a human-readable notation for
phylogenetic trees; a tree with tree taxa, $A$, $B$, and $C$, is
written like this
\begin{center}
\texttt{((A,B),C);}
\end{center}
The leaves are denoted by their labels, while internal nodes are
denoted by round brackets; the root node is followed by a
semi-colon. When drawn with \texttt{new2view} it looks like in
Figure~\ref{fig:phy}A. This tree contains no explicit branch lengths,
so all branches are drawn with length 1. Branch lengths can be added
by writing a colon to the right of a node marker, followed by the
distance from that node to its parent, for example,
\begin{center}
  \texttt{((A:0.2,B:0.25):1,C:1.2);}
\end{center}
\texttt{New2view} converts this to Figure~\ref{fig:phy}B.

In addition to distances, individual nodes can also be labeled, for
example with support values. Say the parent node of $A$ and $B$ in our
example tree has support value 98, then we write this in square
brackets behind the node marker
\begin{center}
  \texttt{((A:0.2,B:0.25)[98]:1,C:1.2);}
\end{center}
which \texttt{new2view} converts to Figure~\ref{fig:phy}C.
\begin{figure}
  \begin{center}
    \begin{tabular}{ccc}
      \textbf{A} & \textbf{B} & \textbf{C}\\
\texttt{((A,B),C);} & \texttt{((A:0.2,B:0.25):1,C:1.2);} & \texttt{((A:0.2,B:0.25)[98]:1,C:1.2);}\\\\
\begin{pspicture}(0.000,0.000)(3.000,3.167)
\psline(2.250000,2.866667)(3.000000,2.866667)
\rput(2.625000,3.066667){0.5}
\rput(3.000000,0.000000){\rnode{n1}{}}\uput{4pt}[0.000000]{0.000000}(n1){A}
\rput(3.000000,1.333333){\rnode{n3}{}}\uput{4pt}[0.000000]{0.000000}(n3){B}
\rput(1.500000,0.666667){\rnode{n2}{}}
\rput(1.500000,2.666667){\rnode{n5}{}}\uput{4pt}[0.000000]{0.000000}(n5){C}
\rput(0.000000,1.666667){\rnode{n4}{}}
\ncangle[angleB=180,angleA=-90,armB=0]{n2}{n1}
\ncangle[angleB=180,angleA=90,armB=0]{n2}{n3}
\ncangle[angleB=180,angleA=-90,armB=0]{n4}{n2}
\ncangle[angleB=180,angleA=90,armB=0]{n4}{n5}
\end{pspicture}
 & \begin{pspicture}(0.000,0.000)(3.000,3.167)
\psline(1.800000,2.866667)(3.000000,2.866667)
\rput(2.400000,3.066667){0.5}
\rput(2.880000,0.000000){\rnode{n1}{}}\uput{4pt}[0.000000]{0.000000}(n1){A}
\rput(3.000000,1.333333){\rnode{n3}{}}\uput{4pt}[0.000000]{0.000000}(n3){B}
\rput(2.400000,0.666667){\rnode{n2}{}}
\rput(2.880000,2.666667){\rnode{n5}{}}\uput{4pt}[0.000000]{0.000000}(n5){C}
\rput(0.000000,1.666667){\rnode{n4}{}}
\ncangle[angleB=180,angleA=-90,armB=0]{n2}{n1}
\ncangle[angleB=180,angleA=90,armB=0]{n2}{n3}
\ncangle[angleB=180,angleA=-90,armB=0]{n4}{n2}
\ncangle[angleB=180,angleA=90,armB=0]{n4}{n5}
\end{pspicture}
 & \begin{pspicture}(0.000,0.000)(3.000,3.167)
\psline(1.800000,2.866667)(3.000000,2.866667)
\rput(2.400000,3.066667){0.5}
\rput(2.880000,0.000000){\rnode{n1}{}}\uput{4pt}[0.000000]{0.000000}(n1){A}
\rput(3.000000,1.333333){\rnode{n3}{}}\uput{4pt}[0.000000]{0.000000}(n3){B}
\rput(2.400000,0.666667){\rnode{n2}{}}\uput{4pt}[135.000000]{0.000000}(n2){\small 98}
\rput(2.880000,2.666667){\rnode{n5}{}}\uput{4pt}[0.000000]{0.000000}(n5){C}
\rput(0.000000,1.666667){\rnode{n4}{}}
\ncangle[angleB=180,angleA=-90,armB=0]{n2}{n1}
\ncangle[angleB=180,angleA=90,armB=0]{n2}{n3}
\ncangle[angleB=180,angleA=-90,armB=0]{n4}{n2}
\ncangle[angleB=180,angleA=90,armB=0]{n4}{n5}
\end{pspicture}

    \end{tabular}
  \end{center}
  \caption{Three example trees drawn with
    \texttt{new2view}. (\textbf{A}) has no explicit branch lengths, while
    (\textbf{B}) does have them; (\textbf{C}) differs from
    (\textbf{B}) by the addition of the support value \emph{98}.}\label{fig:phy}
\end{figure}

\section{Getting Started}
The program \texttt{new2view} was written in C on a computer running Linux.
Please contact \texttt{haubold@evolbio.mpg.de\ignorespaces
} if there are any problems
with the program.
\begin{itemize}
\item Obtain the package\\
\texttt{git clone https://www.github.com/evolbioinf\ignorespaces
/new2view}
\item Change into the directory just downloaded
\begin{verbatim}
cd new2view
\end{verbatim}
and make \texttt{new2view}
\begin{verbatim}
make
\end{verbatim}
\item Test \texttt{new2view}
\begin{verbatim}
make test
\end{verbatim}
\item The executable \texttt{new2view} is located in the
  directory \texttt{build}. Place it into your \texttt{PATH}.
\item Make the documentation
\begin{verbatim}
make doc
\end{verbatim}
This calls the typesetting program \texttt{latex}, so please make sure
it is installed before making the documentation. The typeset manual is
located in
\begin{verbatim}
doc/new2view.pdf
\end{verbatim}
\end{itemize}

\section{Change Log}
Please use
\begin{verbatim}
git log
\end{verbatim}
to list the change history.

%% \bibliography{ref}
\end{document}

